\documentclass[conference]{IEEEtran}
\usepackage[top=3cm, bottom=2cm, left=2cm, right=2cm, columnsep=20pt]{geometry}
\usepackage{pdfpages}
\usepackage{graphicx}
\usepackage{etoolbox}
\apptocmd{\sloppy}{\hbadness 10000\relax}{}{}
% \usepackage[numbers]{natbib}
\usepackage[T1]{fontenc}
\usepackage{ragged2e}
\usepackage[french]{babel}
\usepackage{listings}
\usepackage{color}
\usepackage{soul}
\usepackage[utf8]{inputenc}
\usepackage[export]{adjustbox}
\usepackage{mathrsfs,amsmath} 
\usepackage{amssymb}
\usepackage{float}
\usepackage{csquotes}
\usepackage{fancyhdr}
\usepackage{wallpaper}
\usepackage{siunitx}
\usepackage[indent]{parskip}
\usepackage{textcomp}
\usepackage{gensymb}
\usepackage{multirow}
\usepackage[hidelinks]{hyperref}
\usepackage{abstract}
\usepackage{tabularx}

\usepackage{xcolor}
\definecolor{codegreen}{rgb}{0,0.6,0}
\definecolor{codegray}{rgb}{0.5,0.5,0.5}
\definecolor{codepurple}{rgb}{0.58,0,0.82}
\definecolor{backcolour}{rgb}{0.95,0.95,0.92}
\lstdefinestyle{mystyle}{
    backgroundcolor=\color{backcolour},   
    commentstyle=\color{codegreen},
    keywordstyle=\color{magenta},
    numberstyle=\tiny\color{codegray},
    stringstyle=\color{codepurple},
    basicstyle=\ttfamily\footnotesize,
    breakatwhitespace=false,         
    breaklines=true,                 
    captionpos=b,                    
    keepspaces=true,                 
    numbers=left,                    
    numbersep=5pt,                  
    showspaces=false,                
    showstringspaces=false,
    showtabs=false,                  
    tabsize=2
}
\lstset{style=mystyle}


%----------------------------------------------------

\setlength{\parindent}{0pt}
\newcolumntype{Y}[1]{>{\Centering\hspace{0pt}\hsize=#1\hsize}X}
\newcommand{\inlinecode}{\normalfont\texttt}
\usepackage{enumitem}
\setlist[itemize]{label=\textbullet}

\begin{document}

%----------------------------------------------------
\title{Spectromètre\\
\large Travail préparatoire \\
PHS3910 -- Techniques expérimentales et instrumentation\\ 
Équipe L3}

\author{\IEEEauthorblockN{Émile Guertin-Picard}
\IEEEauthorblockA{2208363}
\and
\IEEEauthorblockN{Maxime Rouillon}
\IEEEauthorblockA{2213291}
\and
\IEEEauthorblockN{Marie-Lou Dessureault}
\IEEEauthorblockA{2211129}
\and
\IEEEauthorblockN{Philippine Beaubois}
\IEEEauthorblockA{2211153}
}

\maketitle

\textit{\textbf{Résumé} -- yap yap}

\section{Introduction}
yap yap

\section{Méthodes \label{methodes}}
Le parcours d'un faisceau passant à travers le spectromètre a ensuite été simulé à l'aide 
de l'optique de Fourier. Le champ initial a été modélisé par une fonction rectangle en deux dimensions,
afin de modéliser la forme du champ après avoir traversé une fente. Celui-ci peut donc être décrit par:
\[U(x_0,y_0)=rect(\frac{x_0}{a})rect(\frac{y_0}{b}),\]
où $a$ et $b$ sont la largeur et la hauteur de la fente respectivement. Le champ transmis par 
la première lentille a donc pu être déterminé:
\begin{align*}
    U_1(x_1,y_1)&\propto\mathscr{F}\left\{U(x_0,y_0)\right\}(\frac{x_1}{\lambda f_1},\frac{y_1}{\lambda f_1})\\
    &\propto \mathscr{F}\left\{rect(\frac{x_0}{a})rect(\frac{y_0}{b})\right\}(\frac{x_1}{\lambda f_1},\frac{y_1}{\lambda f_1})\\
    &\propto sinc(\frac{x_1a}{\lambda f_1})\ast sinc(\frac{y_1b}{\lambda f_1}),
\end{align*}
où $f_1$ est la longueur focale de la lentille. Le réseau de diffraction blazé modifie la forme du champ, ce qui a pu être
modélisé à l'aide d'un masque:
\[M(x_1,y_1)=(comb(\frac{x_1}{\Lambda})\ast rect(\frac{x_1}{\Lambda})e^{i\beta x})rect(\frac{x_1}{N\Lambda}).\]
Le champ devient donc:
\[U_1(x_1,y_1)\propto sinc(\frac{x_1a}{\lambda f_1})\ast sinc(\frac{y_1b}{\lambda f_1}) M(x_1,y_1).\]
Le champ à la sortie de la deuxième lentille revient donc à faire la transformée de Fourier
de $U_1(x_1,y_1)$:

\footnotesize
\begin{align*}
    U_2(x_2,y_2)\propto&\ \mathscr{F}\left\{sinc(\frac{x_1a}{\lambda f_1})\ast sinc(\frac{y_1b}{\lambda f_1})M(x_1,y_1)\right\}(\frac{x_2}{\lambda f_2},\frac{y_2}{\lambda f_2})\\
    \propto&\ rect(\frac{x_2 f_1}{a f_2})rect(\frac{y_2 f_1}{b f_2})\ast comb(\frac{x_2 \Lambda}{\lambda f_2})\\
    &\ sinc(\frac{x_2 \Lambda}{\lambda f_2})\ast \delta(2\pi(x_2-\frac{\beta}{2\pi}))\ast sinc(\frac{x_2 N\Lambda}{\lambda f_2})
\end{align*}
\normalsize 

\section{Résultats \label{resultats}}
yap yap

\section{Discussion}
yap yap

\section{Annexes}

\subsection{Développement mathématique détaillé}
yap yap


\clearpage

% \bibliographystyle{unsrtnat}
% \bibliography{My_Library}

\end{document}
